\documentclass{article}

\begin{document}
\section{ABA problem}

To solve the ABA problem the bit-stealing method was chosen.
On a 64-bit machine only 48\,bits are used for the address.
The remaining 15 bits are used as a timestamp and 1 bit is used to mark the node to request it's deletion.
The timestamp is a counter, which is incremented on every deletion. Two solutions are shown using pseudocodes of the find function.

\subsection{Method\,1}

In this method a tagged pointer contains the tag of the current node and a pointer to the next node.

\begin{enumerate}
\item Get the timestamp of current node from current node next field
\item Get successor and value from current node
\item Compare timestamp of current node with the timestamp acquired at step\,1
\item Compare predecessor next pointer with the current node pointer
\item Set predecessor to current
\item Set current to successor
\end{enumerate}

Here step\,3 is responsible to make sure, that the value of the current node has not been changed since we acquired it at step\,2.
Consider the case, that thread\,\texttt{A} loaded atomically the pointer and the timestamp of the current node.
Thread\,\texttt{B} deletes the current node and adds a new value reusing the node.
Now thread A will read a different value, but since the timestamp is different in the current node, then which was loaded earlier, the algorithm restarts.

Step\,4 is responsible to check that the predecessor still holds the current as the next node.
Consider the case, that thread\,\texttt{A} loaded atomically the pointer to the successor.
Thread \,\texttt{B} deletes the successor node and adds a new value reusing the node.
Now the successor has a different position in the list.
Now thread\,\texttt{A} continues and sets current to successor.
If thread\,\texttt{A} would continue with the next successor, it would skip elements from the list.
Therefore the predecessors next pointer is compared to the current pointer, if they are not the same, then the algorithm restarts.

\subsection{Method\,2}

In this method a tagged pointer contains the tag of the next node and a pointer to the next node.

\begin{enumerate}
\item Get tagged pointer to successor from current node
\item Get value from current node
\item Compare predecessor next tagged pointer with current tagged pointer
\item Set predecessor to current
\item Set current to successor
\end{enumerate}

Here step\,3 is responsible to check if the current node is still at the same position and has the same timestamp.
The reasons are the same as in the first method, but here the two comparison is done in one.

\subsection{Comparison of the two methods}

Although method\,2 contains the tag and pointer to a node in the same tagged pointer and only one comparison is needed, but the tag is accessible only indirectly from the predecessor.
When deleting from a skiplist, each level is being marked, so it is faster to mark them directly, instead of restarting if the predecessor has been changed. 

\section{Memory reclamation}

When deleting a node physically from the list, the memory cannot be freed, since other threads can have an unfinished operation on the node.
For this reason the deleted nodes are stored in a thread local LIFO.
A node from a skiplist can be pushed to the LIFO, when it is physically removed from the bottom level.
If the thread local LIFO contains an empty node, then it is reused when a new value is added.
When nodes in a skiplist are reused, the number of next pointers must be the number of maximum levels, because the new value might get a higher level, then the previous one had.

\end{document}